\documentclass[]{article}
\usepackage[T1]{fontenc}
\usepackage[utf8]{inputenc}
\usepackage{hyperref}
\usepackage{mathtools}

\usepackage{titlesec}
\usepackage{float}

\setcounter{secnumdepth}{4}
\titleformat{\paragraph}
{\normalfont\normalsize\bfseries}{\theparagraph}{1em}{}
\titlespacing*{\paragraph}
{0pt}{3.25ex plus 1ex minus .2ex}{1.5ex plus .2ex}

\setlength{\parindent}{0pt}

\usepackage{amsmath , amsfonts, amssymb}

\begin{document}

\tableofcontents

\section{Introduction} 

\section{Modélisation}
\subsection{Spécification du modèle}
\subsubsection{Choix des variables}

\subsubsection{Spécification du modèle}
Notre premier modèle est formulé de la sorte : 
\begin{equation}
\begin{split}
    locataire_i = & \beta_1 + \beta_2diplomesup_i + \beta_3jeunes_i + \beta_4persagees_i\\
    + & \beta_5appartements_i + \beta_6chomage_i+ \beta_7urbanisation_i + \beta_8pauvrete_i + \varepsilon_i 
\end{split}
\end{equation}
\subsection{Première Estimation du modèle}
\subsubsection{Résultats et interprétation}
On estime par la méthode des moindres carrés ordinaires l'équation. On obtient les résultats suivants :
\begin{table}
    \centering
    \begin{tabular}{|c|c|c|}
    \hline
    Variables      & \begin{tabular}{c}Coefficients \\estimés\end{tabular} & Écart-type   \\
    \hline
    $diplomesup$   & $0.4028115$                                                      & $0.0844738$ \\
    \hline
    $jeunes$       & $0.9092502$                                                      & $0.285378$  \\
    \hline
    $persagees$    & $0.1280139$                                                      & $0.2338338$ \\
    \hline
    $appartements$ & $0.1638682$                                                      & $0.0307458$ \\
    \hline
    $chomage$      & $-0.1978747$                                                     & $0.2883803$ \\
    \hline
    $urbanisation$ & $0.0080034$                                                      & $0.028728$ \\
    \hline
    $pauvrete$     & $0.7533171$                                                      & $0.1317305$ \\
    \hline
    $constante$    & $-10.84531$                                                      & $13.04414$ \\
    \hline
    \end{tabular}
\end{table}

\begin{equation*}
    \begin{split}
            \hat{locataires}_i &= -10.845 + 0.403 \times diplomesup_i \\
            &+ 0.909 \times jeunes_i + 0.128 \times persagees_i \\
            &+ 0.164 \times appartements_i + -0.198 \times chomage_i \\
            &+ 0.008 \times urbanisation_i + 0.753 \times pauvrete_i
    \end{split}
\end{equation*}
Avec : 
\begin{equation*}
    N = 96 \; K = 8 \quad  SCE = 3873,9635 \quad SCR = 485,248588 \quad SCT = 4359,20494
\end{equation*}
On peut calculer le coefficient de détermination $R^{2}$ :
\begin{equation*}
    R^{2} = \frac{SCE}{SCT} = \frac{3873,9635}{4359,20494} = 0,8887
\end{equation*}
On calcule également le coefficient de détermination ajusté aux nombres de variables :
\begin{equation*}
    \bar{R}^{2} = 1 - \frac{SCR/(N-K)}{SCT/(N-1)} = \frac{485,248588/88}{4359,20494/95} = 0,8798
\end{equation*}
Le coefficient de détermination est élevé $R^{2} = 0.89$, cela laisse penser que notre la combinaison linéaires de nos variables expliquent bien la part de locataires par
départements. On effectue donc les tests de significativité des paramètres et de significativité conjointe pour prouver cela.
\subsubsection{Tests de significativité}
Dans un premier temps, nous allons effectuer un est $F$ de significativité conjointe de tous les paramètres de la pente : 
\begin{equation*}
\begin{split}
    H_0 &: \beta_2 = \beta_3 = \dots \beta_K =0 \\
    H_1 &: \text{Au moins un }\beta_j \neq 0.
\end{split}
\end{equation*}
La statistique de test sous l'hypothèse nulle est distribuée selon une loi $F$ de Fisher :
\begin{equation*}
    F = \frac{SCE/(K-1)}{SCR/(N-K)} \sim F(K-1, N-K)
\end{equation*}
On choisi un niveau de test à $\alpha = 5\%$. Et l'on compare la statistique calculée au quantile à $95\%$ de la distribution $F$ de Fisher avec respectivement $7$ et $88$ degrés de liberté
au numérateur et au dénominateur. 
\begin{equation*}
    F_{1-\alpha} (K-1, N-K) = F_{0.95}(7, 88) = 2.121
\end{equation*}
On cherche maintenant à savoir si les paramètres estimés sont significativement différents de 0. Pour cela on effectue le test suivant sur les $j = 8$ variables :
\begin{equation*}
\begin{split}
    H_0 &: \beta_j =0 \\
    H_1 &: \beta_j \neq 0
\end{split}
\end{equation*}
La statistique de test calculée sous l'hypothèse nulle est distribuée selon une loi $t$ de Student : 
\begin{equation*}
    t_{\beta_j} = \frac{\hat{\beta_j}}{s_{\hat{\beta_j}}} \sim t(N-K)
\end{equation*}
On se place à un niveau de test bilatéral de $\alpha = 5\%$ et la statistique en valeur absolue doit être comparée au quantile à $97,5\%$ de la distribution $t$ de Student à $N-K = 96 - 8 = 
88$ degrés de liberté, soit le seuil critique :
\begin{equation*}
    t_{1-\alpha/2}(N-K) = t_{0,975}(88) = 1,987289865 
\end{equation*}

Règle de décision : 
\begin{align*}
    &\text{Si} |t_{\beta_j}| < 1.99 & &\text{On accepte } H_{0}\\
    &\text{Si} |t_{\beta_j}| \geq 1.99 & &\text{On rejette } H_{0}
\end{align*}
Tous calculs faits, on obtient les résultats suivants :
\begin{table}[H]
    \centering
    \begin{tabular}{|c|c|} 
    \hline
    Variables      & t calculé  \\ 
    \hline
    $constante$    & $t_{\beta_1} =-0.83$      \\
    \hline
    $diplomesup$   & $t_{\beta_2} = 4.77$       \\ 
    \hline
    $jeunes$       & $t_{\beta_3} =3.19$       \\ 
    \hline
    $persagees$    & $t_{\beta_4} =0.55$       \\ 
    \hline
    $appartements$ & $t_{\beta_5} =5.33$       \\ 
    \hline
    $chomage$      & $t_{\beta_6} =-0.69$      \\ 
    \hline
    $urbanisation$ & $t_{\beta_7} =0.28$       \\ 
    \hline
    $pauvrete$     & $t_{\beta_8} =5.72$       \\ 
    \hline
    \end{tabular}
\end{table}
Les statistiques de test $t_{\beta_1},t_{\beta_7},t_{\beta_6} et t_{\beta_3}$ sont inférieures en valeur en absolue au seuil. On accepte donc l'hypothèse nulle,
les paramètres ne sont pas significatifs, on va donc retirer les variables $urbanisation$, $chomage$ et $persagees$.

En revanche, $t_{\beta_2}, t_{\beta_3}, t_{\beta_5}, t_{\beta_8}$ sont supérieures en valeur absolue au seuil critique de $1,99$. On rejette l'hypothèse nulle pour ces paramètres
ils sont significatifs (différents de zéro), les variables $diplomesup$, $jeunes$, $appartements$ et $pauvrete$ permettent d'expliquer en partie $locataire$
\subsubsection{Etude de la multicolinéarité}

\subsection{Estimation du modèle final}
\subsubsection{Interprétation des résultats}
\subsubsection{Tests de significativité}
\subsubsection{Etude de la normalité des résidus}
\subsubsection{Test sur l'hétéroscédasticité}
\subsubsection{Estimation du modèle avec écart-type de White}
\section{Conclusion}



\end{document}
