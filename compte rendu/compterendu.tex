\documentclass[]{article}
\usepackage[T1]{fontenc}
\usepackage[utf8]{inputenc}
\usepackage{hyperref}
\usepackage{mathtools}

\usepackage{titlesec}
\usepackage{float}
\usepackage{tabularx,booktabs,caption,ragged2e}

\setcounter{secnumdepth}{4}
\titleformat{\paragraph}
{\normalfont\normalsize\bfseries}{\theparagraph}{1em}{}
\titlespacing*{\paragraph}
{0pt}{3.25ex plus 1ex minus .2ex}{1.5ex plus .2ex}
\setlength{\parindent}{0pt}

\captionsetup{skip=0.333\baselineskip}


\usepackage{amsmath , amsfonts, amssymb}

\begin{document}

\tableofcontents

\section{Introduction} 

\newpage
\section{Modélisation}
\subsection{Spécification du modèle}
\subsubsection{Choix des variables}
\subsubsection{Spécification du modèle}
Le premier modèle est formulé de la sorte : 
\begin{equation}
\begin{split}
		locataire_i =  \beta_1 &+ \beta_2diplomesup_i + \beta_3jeunes_i + \beta_4persagees_i \\
						&+ \beta_5appartements_i + \beta_6chomage_i + \beta_7urbanisation_i \\
						&+ \beta_8pauvrete_i + \varepsilon_i 
\end{split}
\end{equation}
\subsection{Première Estimation du modèle}
\subsubsection{Résultats et interprétation}
On estime par la méthode des moindres carrés ordinaires les coefficients de l'équation précédente.
\begin{table}[H]
\centering
\begin{tabular}{l*{1}{cc}}
\toprule
            &\multicolumn{2}{c}{Taux locataire}\\
Variable            & Coefficient         &  Ecart-Type\\
\midrule
Constante & -10.84531 & 13.04414\\
part\_diplome\_sup	& 0,4028115	& 0,0844738\\
part\_jeune  & 0,9092502 & 0,285378\\
part\_soixante\_cinq\_plus& 0,1280139 & 0,2338338\\
part\_appartement& 0,1638682 & 0,0307458\\
taux\_chomage& 0,1978747 & 0,2883803\\
taux\_urba   & 0,0080034 & 0,028728\\
taux\_pauvrete& 0,7533171& 0,1317305\\
\midrule
Nb. observations&          96         &            \\
SCE & $3873,9635$ \\
SCR & $485,248588$ \\
SCT & $4359,20494$ \\
\bottomrule
\end{tabular}
\end{table}
Le modèle s'écrit alors : 
\begin{equation*}
    \begin{split}
			\hat{locataire}_i = -10.845 &+ 0.403 \times diplomesup_i \\
            &+ 0.909 \times jeunes_i \\
            &+ 0.128 \times persagees_i \\
            &+ 0.164 \times appartements_i \\
            &+ 0.198 \times chomage_i \\
            &+ 0.008 \times urbanisation_i \\
            &+ 0.753 \times pauvrete_i \\
    \end{split}
\end{equation*}
On peut calculer le coefficient de détermination $R^{2}$ :
\begin{equation*}
    R^{2} = \frac{SCE}{SCT} = \frac{3873,9635}{4359,20494} = 0,8887
\end{equation*}
On calcule également le coefficient de détermination ajusté aux nombres de variables :
\begin{equation*}
    \bar{R}^{2} = 1 - \frac{SCR/(N-K)}{SCT/(N-1)} = \frac{485,248588/88}{4359,20494/95} = 0,8798
\end{equation*}
Les deux coefficients sont élevés, cela laisse donc à supposer que l'ajustement de de la regression est de bonne qualité. Cependant, des tests de 
significativité (des paramètres et conjointe), et une étude de la multicolinéarité restent à être menés pour confirmer cela.
\subsubsection{Tests de significativité}
Dans un premier temps, on réalise un test $F$ de significativité conjointe de tous les paramètres. L'hypothèse nulle, et l'hypothèse alternative de ce test sont telles que :
\begin{equation*}
\begin{split}
    H_0 &: \beta_2 = \beta_3 = \dots \beta_K =0 \\
    H_1 &: \text{Au moins un }\beta_j \neq 0.
\end{split}
\end{equation*}
La statistique de test sous l'hypothèse nulle est distribuée selon une loi $F$ de Fisher :
\begin{equation*}
    F = \frac{SCE/(K-1)}{SCR/(N-K)} \sim F(K-1, N-K)
\end{equation*}
On choisi un niveau de test à $\alpha = 5\%$. Et l'on compare la statistique calculée au quantile à $95\%$ de la distribution $F$ de Fisher avec 
comme degrés de liberte $7$ et $88$ respectivement au numérateur et au dénominateur. 
\begin{equation*}
    F_{1-\alpha} (K-1, N-K) = F_{0.95}(7, 88) = 2.121
\end{equation*}
Après calculs, $F = 100,364$. La statisitique est supérieure au seuil, l'hypothèse nulle est rejetée au moins une variable permet d'expliquer le modèle.
\\ \\
On cherche maintenant à déterminer plus précisément quels sont les paramètres estimés significativement différents de 0. Pour cela, un test $t$ de 
significativité est effectué sur sur les $8$ paramètres :
\begin{equation*}
\begin{split}
    H_0 &: \beta_j =0 \\
    H_1 &: \beta_j \neq 0
\end{split}
\end{equation*}
La statistique de test calculée sous l'hypothèse nulle est distribuée selon une loi $t$ de Student : 
\begin{equation*}
    t_{\beta_j} = \frac{\hat{\beta_j}}{s_{\hat{\beta_j}}} \sim t(N-K)
\end{equation*}
On se place à un niveau de test bilatéral de $\alpha = 5\%$ et la statistique en valeur absolue doit être comparée au quantile à $97,5\%$ de la distributi
on $t$ de Student à $88$ degrés de liberté, soit le seuil critique :
\begin{equation*}
    t_{1-\alpha/2}(N-K) = t_{0,975}(88) = 1,987289865 
\end{equation*}
Tous calculs faits, on obtient les résultats suivants :
\begin{table}[H]
\centering
\begin{tabular}{l*{1}{c}}
\toprule
            &Statistique t\\
\midrule
constante      &   -.8314317\\
part\_diplome\_sup&    4.768476\\
part\_jeune  &    3.186125\\
part\_soixante\_cinq\_plus&     .547457\\
part\_appartement&    5.329783\\
taux\_chomage&   -.6861589\\
taux\_urba   &    .2785922\\
taux\_pauvrete&    5.718624\\
\bottomrule
\end{tabular}
\end{table}
Les statistiques de test $t_{\beta_1},t_{\beta_7},t_{\beta_6}$ et $t_{\beta_3}$ sont inférieures en valeur en absolue au seuil. L'hypothèse nulle, est
acceptée pour ces paramètres, ils ne sont pas significatifs. Les variables $urbanisation$, $chomage$ et $persagees$ seront donc retirées.

En revanche, $t_{\beta_2}, t_{\beta_3}, t_{\beta_5}, t_{\beta_8}$ sont supérieures en valeur absolue au seuil critique de $1,99$. On rejette l'hypothèse nulle pour ces paramètres
ils sont significatifs, les variables $diplomesup$, $jeunes$, $appartements$ et $pauvrete$ permettent d'expliquer en partie $locataire$
\subsubsection{Etude de la multicolinéarité}
Une corrélation forte entre plusieurs variables explicatives peut induire une présence de multicolinéarité dans le modèle. Un calcul du VIF est fait pour chacune des variables.
\begin{table}[H]
\centering
\caption{Facteur d'inflation de la variance}
\begin{tabular}{l*{1}{c}}
\toprule
Variable            &         VIF\\
\midrule
constante & 0 \\
part\_diplome\_sup&    3,199862\\
part\_jeune  &    12,48724\\
part\_soixante\_cinq\_plus&    14,58855\\
part\_appartement&    4,848903\\
taux\_chomage&    3,094149\\
taux\_urba   &    4,416773\\
taux\_pauvrete&    2,663816\\
\bottomrule
\end{tabular}
\end{table}
Il y a donc en effet de la multicolinéarité dans le modèle, les variables $jeune$, $persagees$ et $appartements$ ont un vif
élevé, elles seront donc retirées du modèle afin de traiter la multicolinéarité.
\subsection{Estimation du modèle final}
Un second modèle est formulé à la suite des résultats précédents. Ce modèle est tel que :
\begin{equation*}
    \begin{split}
            locataire_i =  \beta_1 &+ \beta_2diplomesup_i + \beta_3jeunes_i + \beta_5pauvrete_i + \varepsilon_i 
    \end{split}
\end{equation*}
Les paramètres du modèle sont estimés grace aux MCO.
\begin{table}[H]
\centering
\caption{}
\label{table:secondeReg}
\begin{tabular}{l*{1}{ccc}}
\toprule
Variable            & Coefficient&  Ecart-type&Statistique t\\
\midrule
diplomesup&    0,8001962&    0,0612541&    13,06355\\
jeunes  &     0,960763&    0,1044884&    9,194929\\
pauvrete&    0,8518716&     0,096727&    8,806969\\
constante      &   -9,171721&    3,132582&   -2,927847\\
\midrule
$N$       &          96& SCE           &   3634,97216           \\
$R^{2}$ & 0,8339 & SCR & 724,232776   \\ 
$\bar{R}^2$ & 0,8284 & SCT & 4359,20494\\ 
\bottomrule
\end{tabular}
\end{table}
\subsubsection{Interprétation des résultats}
Les coefficients de détermination simple et ajusté sont relativement élevés, la qualité de l'ajustement du modèle est donc bonne. En théorie la
multicolinéarité a été traitée, pour confirmer cela, un calcul des VIF est efféctué.
\begin{table}[H]
\centering
\caption{VIF}
\begin{tabular}{l*{1}{c}}
\toprule
Variable            &         VIF\\
\midrule
constante & 0 \\
diplomesup & 1,18 \\
jeunes & 1,17 \\
pauvrete & 1,01 \\
\bottomrule
\end{tabular}
\end{table}
Les trois variables présentes dans le modèle ont un VIF proche de un, elles ne génèrent quasiement pas de multicolinéarité.
\subsubsection{Tests de significativité}
Des tests $t$ de siginificativité des paramètres fait pour déterminer si les paramètres estimés sont significatifs. Le test est le même que dans la
section 2.2.2. Les statistiques $t$ ont été calculées lors et sont dans le tableau ~\ref{table:secondeReg}.\\

Un niveau de test bilatéral de $\alpha = 5 \%$ est chosi, la statistique  $t$ doit donc être comparée au quantile à  $97,5\%$ de la distribution de
Student à 92 degrés de liberté. Le seuil critique est donc :
\begin{equation*}
t_{0,975}(92) = 1,98608631695112 
\end{equation*}
Toutes les statistiques $t$ des paramètres de la regression sont supérieurs en valeur absolue au seuil critique. $H_0$ est rejetée dans tous les cas.
Tous les coefficients sont significativements différents de 0.
\\ \\
Afin de conclu
\subsubsection{Etude de la normalité des résidus}
\subsubsection{Test sur l'hétéroscédasticité}
\subsubsection{Estimation du modèle avec écart-type de White}
\section{Conclusion}

dico : VIF = Variance inflation factor
MCO = moindres carrés ordinaires

\end{document}
